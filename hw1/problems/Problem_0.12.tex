The following illustration only holds for \textit{simple} graphs, that is graphs without loops. In the trivial case, the required principle is violated by a graph of two vertices, one loop, and no edges. \\

\noindent For any graph $G$ with $ g $ nodes, the lowest possible degree of a node $ n $ is 0 (a node that is not connected to any other node) and the highest possible degree of a node $ n $ is $g-1$ (a node that is connected to every other node). Thus, the set of possible degrees of nodes in $G$ is \setof{0, 1, 2, \dots g-1}. This is a set of size $g$. However, for any given graph, there cannot simultaneously be a node of degree $ 0 $ and degree $ g-1 $, as that would be a contradiction: There cannot simultaneously be a node that connects to all other nodes and a node that does not connect to all other nodes in the same graph. Thus the \textit{true} size of possible degrees of $ G $ is $g-1$, as it will contain \textit{either} a node of degree $0$ or of degree $g-1$, but not both. Since $G$ has $g$ vertices and only $g-1$ possible values of $\degree{n}$ for all vertices $n$, by the pigeonhole principle, there must be at least two vertices with the same degree.